\documentclass[11pt,a4paper]{article}
\usepackage[utf8]{inputenc}
\usepackage[T1]{fontenc}
\usepackage[polish]{babel}
\usepackage{lmodern}
\usepackage{graphicx}
\usepackage{epstopdf}
\usepackage{anysize}
\usepackage{makeidx}
\usepackage{hyperref}
\usepackage{listings}



\makeatletter
\renewcommand{\maketitle}{
\begin{titlepage}
\begin{center}

\LARGE{AKADEMIA GÓRNICZO-HUTNICZA}

\vspace*{1cm}
\includegraphics[scale=1.8]{agh.eps}
\vspace*{1cm}

\LARGE{im. Stanisława Staszica w Krakowie}

\rule{\textwidth}{0.4mm}
\LARGE \textsc{\@title}
\rule{\textwidth}{0.4mm}

\vspace*{5mm}


\large
\emph{Autorzy:}\\
Tomasz \textsc{Czarnik}\\
Krzysztof \textsc{Garcarz}\\
Krzysztof \textsc{Śmiłek}\\

\vfill
\vspace*{\stretch{8}}
\rule{\textwidth}{0.4mm}

\large{Wydział Elektroniki, Automatyki, Informatyki i Elektrotechniki}\\
\large{Katedra Automatyki}\\
\large{Laboratorium Biocybernetyki}\\
\vspace*{\stretch{7}}
\@date

\end{center}

\end{titlepage}
}
\makeatother

\title{Projekt sieci LAN dla szkoły średniej}
\date{\today}

\makeindex

\begin{document}

\maketitle

\newpage

\tableofcontents

\newpage

\section{Wstęp}
\subsection{Cel projektu}
Naszym celem jest stworzenie projektu sieci LAN na potrzeby szkoły średniej przy pomocy technologii poznanych na zajęciach z przedmiotu Administrowanie sieciami lokalnymi. Staraliśmy się użyć, w rozsądnych granicach jak najwięcej poznanych narzędzi.
\subsection{Założenia projektu}
Przyjęliśmy następujące założenia odnośnie sieci:
\begin{itemize}
\item Szkoła posiada dwie pracownie komputerowe na parterze i na pierwszym piętrze, każda z nich liczy po 12 komputerów
\item Dodatkowo istnieje potrzeba podłączenia do sieci sekretariatu, gabinetu dyrektora oraz komputera w pokoju nauczycielskim
\item W sieci pracuje również serwer WWW
\item Adresy komputerów obu pracowni przydzielane są przez DHCP
\item Sieć będzie działać w topologii rozszerzonej gwiazdy
\end{itemize}
\subsection {Urządzenia i okablowanie}
Przy realizacji naszej sieci skorzystalibyśmy z dwóch routerów oraz paru switchy. Routery zestawione są serialami, natomiast wszystkie pozostałe urządzenia za pomocą portów FastEthernet połączone skrętką (kablem typu UTP kategorii 5).\\
Kablem skrosowanym połączyliśmy urządzenia tego samego typu (Router-Komputer, Switch-Switch) natomiast prostym urządzenia różnego typu (Router-Switch, Switch-Komputer).

\section{Projekt sieci}
\includegraphics[scale=0.6]{siec.png}  

\section{Konfiguracja}

%===========================================================
\subsection{Konfiguracja interfejsów routerów}
{\bf R1:}\\
\noindent
Ustawienie subinterfejsów FastEthernet:
\begin{lstlisting}
R1(config)#interface fastEthernet 0/0.1
R1(config-subif)#ip address 193.0.1.1 255.255.255.0
R1(config-subif)#no shutdown
R1(config-subif)#encapsulation dot1Q 1

R1(config)#interface fastEthernet 0/0.2
R1(config-subif)#ip address 193.0.2.1 255.255.255.0
R1(config-subif)#no shutdown
R1(config-subif)#encapsulation dot1Q 2
\end{lstlisting}

\noindent
{\bf R2:}\\
Ustawienie interfejsów FastEthernet:
\begin{lstlisting}
R2(config)#interface fastEthernet 0/0
R2(config-subif)#ip address 193.0.5.1 255.255.255.0
R2(config-subif)#no shutdown

R2(config)#interface fastEthernet 1/0
R2(config-subif)#ip address 193.0.4.1 255.255.255.0
R2(config-subif)#no shutdown
\end{lstlisting}

%===========================================================
\subsection{Konfiguracja switcha Sw1}
Stworzenie dodatkowego vlana o numerze 2:
\begin{lstlisting}
Sw1(config)#vlan 2
Sw1(config-vlan)#name VLan2
\end{lstlisting}
Stworzenie łącza trunk:
\begin{lstlisting}
Sw1(config)#interface FastEthernet0/1
Sw1(config-if)#switchport mode trunk
Sw1(config-if)# switchport trunk encapsulation dot1q
\end{lstlisting}
Przypisanie portów do danych VLANów (przypisujemy tylko do VLAN 2, bo reszta połączeń jest automatycznie przypisywana do VLAN 1):
\begin{lstlisting}
Sw1(config)#interface FastEthernet1/1
Sw1(config-if)#switchport mode access
Sw1(config-if)#switchport access vlan 2
\end{lstlisting}
%===========================================================
\subsection{Most IPv6}

\noindent
Ustawienie mostka IPv6 między routerami {\bf R1 - R2}
\begin{lstlisting}
R1(config)# ???
R1(config-if)# ???
R1(config-if)#???
\end{lstlisting}

\begin{lstlisting}
R2(config)# ???
R2(config-if)# ???
\end{lstlisting}

%===========================================================
\subsection{Listy ACL}
Listy kontroli dosępu (ACL) mają za zadanie uniemożliwić komunikowanie się pracowni komputerowych z komputerami znajdującymi się w sekretariacie. Na routerze R2 tworzona jest acces lista na której znajduje się adres NAT pracowni komputerowych wraz z odwróconą maską.
\begin{lstlisting}
R2# conf t
R2(config)#acces-list 1 deny 193.0.3.1 0.0.0.255 	
R2(config)# interface FastEthernet 1/0
R2(config-if)#ip access-group 1 in	
R2(config-if)#exit 
\end{lstlisting}

%===========================================================
\subsection{STP}

%===========================================================
\subsection{Routing}
{\bf R1:}\\
\begin{lstlisting}
R1(config)#router ospf 1
R1(config-router)#network 193.0.3.0 0.0.0.255 area 1
R1(config-router)#redistribute rip subnets 
 \end{lstlisting}
{\bf R2:}\\
\begin{lstlisting}
R2(config)#router rip
R2(config-router)#version 2
R2(config-router)#passive-interface FastEthernet0/0
R2(config-router)#network 193.0.4.0
R2(config-router)#network 193.0.5.0
R2(config-router)#no auto-summary

R2(config)#router ospf 1
R2(config-router)#network 193.0.3.0 0.0.0.255 area 1
R2(config-router)#redistribute rip subnets 
 \end{lstlisting}
%===========================================================
\subsection{NAT}

%===========================================================
Chcemy sprawić, aby komputery z obu pracowni były widziane na zewnątrz pod jednym adresem, w tym celu skonfigurujemy NAT.\\
Najpierw upewniamy się, iż adresy sieciowe obu VLANów są odłączone z routringu i komputery z zewnątrz nie są widoczne.\\
Następnie definiujemy na routerze R1, które interfejsy są wewnętrzne a które prowadzą na zewnątrz:
\begin{lstlisting}
R1(config)#interface FastEthernet 0/0.1
R1(config-subif)#ip nat inside

R1(config)#interface FastEthernet 0/0.2
R1(config-subif)#ip nat inside

R1(config)#interface serial 2/0
R1(config-if)#ip nat outside
\end{lstlisting}

Tworzymy access listę opisującą zakres adresów tłumaczonych na granicy i tworzymy translację:
\begin{lstlisting}
R1(config)#access-list 1 permit 193.0.1.0 0.0.0.255
R1(config)#access-list 1 permit 193.0.2.0 0.0.0.255

R1(config)# ip nat inside source list 1 interface Serial 2/0 overload
\end{lstlisting}
Od tej pory komputery wewnątrz NAT widziane są na zewnątrz pod wspólnym adresem (outside - czyli 193.0.3.1).\\
Sprawdziliśmy działanie przez poleceniem ping: z pracowni dało się spingować serwer, natomiast serwer nie mógł pingować komputerów znajdujących się za NATem.
 
%===========================================================

\subsection{DHCP}
Router R1 będzie przydzielał adresy IP dla komputerów obu pracowni poprzez DHCP.\\
Najpierw konfigurujemy pule adresów:
\begin{lstlisting}
R1(config)#ip dhcp pool PracowniaParter
R1(dhcp-config)#network 193.0.1.0 255.255.255.0
R1(dhcp-config)#default-router 193.0.1.1

R1(config)#ip dhcp pool PracowniaPietro
R1(dhcp-config)#network 193.0.2.0 255.255.255.0
R1(dhcp-config)#default-router 193.0.2.1
\end{lstlisting}
Następnie wyłączamy adresy przydzielone statycznie (interfejs routera):
\begin{lstlisting}
R1(config)#ip dhcp excluded-address 193.0.1.1
R1(config)#ip dhcp excluded-address 193.0.2.1
\end{lstlisting}

Sprawdzamy działanie DHCP poleceniem show ip dhcp binding:
\begin{lstlisting}
IP address       Client-ID/              Lease expiration        Type
                 Hardware address
193.0.1.2        00E0.B0DE.39E2           --                     Automatic
193.0.1.3        0009.7CB3.E46D           --                     Automatic
193.0.2.3        0003.E460.873B           --                     Automatic
193.0.2.2        00E0.A305.48C0           --                     Automatic
\end{lstlisting}
Jak widać DHCP działa poprawnie dla obu VLANów (oczywiście w rzeczywistości byłoby po 12 kompóterów na pracownię).
\section{Podsumowanie i wnioski}

\end{document}
