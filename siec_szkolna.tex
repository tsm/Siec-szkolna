\documentclass[11pt,a4paper]{article}
\usepackage[utf8]{inputenc}
\usepackage[T1]{fontenc}
\usepackage[polish]{babel}
\usepackage{lmodern}
\usepackage{graphicx}
\usepackage{epstopdf}
\usepackage{anysize}
\usepackage{makeidx}
\usepackage{hyperref}



\makeatletter
\renewcommand{\maketitle}{
\begin{titlepage}
\begin{center}

\LARGE{AKADEMIA GÓRNICZO-HUTNICZA}

\vspace*{1cm}
\includegraphics[scale=1.8]{agh.eps}
\vspace*{1cm}

\LARGE{im. Stanisława Staszica w Krakowie}

\rule{\textwidth}{0.4mm}
\LARGE \textsc{\@title}
\rule{\textwidth}{0.4mm}

\vspace*{5mm}


\large
\emph{Autorzy:}\\
Tomasz \textsc{Czarnik}\\
Krzysztof \textsc{Garcarz}\\
Krzysztof \textsc{Śmiłek}\\

\vfill
\vspace*{\stretch{8}}
\rule{\textwidth}{0.4mm}

\large{Wydział Elektroniki, Automatyki, Informatyki i Elektrotechniki}\\
\large{Katedra Automatyki}\\
\large{Laboratorium Biocybernetyki}\\
\vspace*{\stretch{7}}
\@date

\end{center}

\end{titlepage}
}
\makeatother

\title{Projekt sieci LAN dla szkoły średniej}
\date{\today}

\makeindex

\begin{document}

\maketitle

\newpage

\tableofcontents

\newpage

\section{Wstęp}
\subsection{Cel projektu}
Naszym celem jest stworzenie projektu sieci LAN na potrzeby szkoły średniej przy pomocy technologii poznanych na zajęciach z przedmiotu Administrowanie sieciami lokalnymi. Staraliśmy się użyć, w rozsądnych granicach jak najwięcej poznanych narzędzi.
\subsection{Założenia projektu}
Przyjęliśmy następujące założenia odnośnie sieci:
\begin{itemize}
\item Szkoła posiada dwie pracownie komputerowe na parterze i na pierwszym piętrze, każda z nich liczy po 12 komputerów
\item Dodatkowo istnieje potrzeba podłączenia do sieci sekretariatu, gabinetu dyrektora oraz komputera w pokoju nauczycielskim
\item W sieci pracuje również serwer WWW
\item Adresy komputerów w całej sieci przydzielane są przez DHCP
\item Sieć będzie działać w topologii rozszerzonej gwiazdy
\end{itemize}
\subsection {Urządzenia i okablowanie}
Przy realizacji naszej sieci skorzystalibyśmy z dwóch routerów oraz paru switchy. Routery zestawione są serialami, natomiast wszystkie pozostałe urządzenia za pomocą portów FastEthernet połączone "skrętką" (kablem typu UTP kategorii 5).\\
Kablem skrosowanym połączyliśmy urządzenia tego samego typu (Router-Komputer, Switch-Switch) natomiast prostym urządzenia różnego typu (Router-Switch, Switch-Komputer).

\section{Projekt sieci}
\includegraphics[scale=0.7]{siec.png}  

\section{Konfiguracja}
\subsection {Konfigurcja routera R1}

\section{Podsumowanie i wnioski}

\end{document}
